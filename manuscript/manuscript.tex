\documentclass[12pt, titlepage]{article}

\usepackage{amsmath}
\usepackage[margin = 1in]{geometry}
\usepackage{graphicx}
\usepackage{booktabs}
\usepackage{natbib}
\usepackage{array}

\usepackage{lipsum}
\usepackage[colorlinks=true, citecolor=blue]{hyperref}


\title{Association between Breast Cancer and Hepatitis C Virus with Age Controlling}
\author{Amy Traianou\\
  Department of Statistics\\
  University of Connecticut
}

\begin{document}
\maketitle


\begin{abstract}
In Egypt Hepatitus C virus infection is a common problem that often impedes treatment
of Breast Cancer in female patients. While previous work has analyze an associaiton 
between HCV seropositivity and breast cancer diagnosis, this paper will focus on the 
potential confounding of age on the association. We will use Fisher's Exact Test to
estimate the crude estimate of the association and the Mantel-Haenszel method to 
stratify the data between ages. The possible effect of age in the association can
lead to improved care for patients. 


\bigskip
\noindent{\sc Keywords}:
Breslow-Day;
Fisher's Exact Test;
Mantel-Haenszel.
\end{abstract}


\section{Introduction}
\label{sec:intro}

This paper will explore the risk factors associated with breast cancer in women. Breast cancer is extremely
prevalent and researchers are constantly trying to determine risk factors to identify women for preventative exams.
There are many studies attempting to determine if there is an association between certain risk factors and 
breast cancer including chronic hepatitis C infection \citep{Larrey2010is} and hepatitis B \citep{vishnu2016does}. 

Specifically, this paper will determine if there is an association between testing positive for hepatitis C virus
and breast cancer while controlling for age in Egyptian populations. Both Hepatitis C and breast 
cancer are extremely prevalent in Egypt \citep{Hussein2021high}. Very recently on October 6th, 
Egypt and Qatar agreed to collaborate in the health sector and use each 
other's expertise. The health ministers specifically mentioned research in Hepatitis C and breast cancer
as a area of interest \citep{arham2022egypt}. Thus, research surrounding these diseases is becoming more
important in the field. 

The rest of the paper is organized into several sections. Section~\ref{sec:data}
describes the study done in 2020 to test if HCV seropositivity is associated with 
breast cancer diagnosis in Egyptian population. Section~\ref{sec:methods} 
describes the following statistical methods; Fisher's Exact Test, Breslow-Day test,
and Mantel-Haenszel method.
Section~\ref{sec:app} applies the methods to data from the study, including
stratafied between ages.
Section~\ref{sec:discuss} concludes the paper.



\section{Data}
\label{sec:data}

The data used in this paper is from a 2020 retrospective case-control 
study based on Egyptian female populations \citep{2020association},
sponsered by Mansoura University in Egypt.

The study group consists of 405 patients treated at the Oncology Center 
- Mansoura University in the past 10 years. In order to be included, 
invasive breast cancer must be biopsy-proven. Patients with unknown 
viral marker status, multiple cancer diagnoses and virus-unrelated 
hepatic pathology were all excluded. 

The control group consists of 
data from a previous study, conducted from 2015-2017. There are 145 
females from the same geographic distribution, all with no previous 
cancer diagnosis. In both groups, all patients are above the age of 
18. For this paper, patients are stratified based on age- older patients 
are considered 45 years or above and younger patients are below 45 years. 
The data is summarized in the following 2x2 contingency tables. 

\vspace{1cm}

2x2 Contingency Table for all patients:

\begin{tabular}{ | m{4cm} | m{4cm}| m{4cm} | m{2cm} | }
  \hline
    & Breast Cancer & No Cancer Diagnosis & Total\\ 
  \hline
  HCV Seropositive & 88 & 15 & 103 \\ 
  \hline
  HCV Seronegative & 317 & 130 & 447 \\ 
  \hline
  Total & 405 & 145 & 550 \\ 
  \hline
\end{tabular}

\vspace{1cm}

2x2 Contingency Table for pateints younger than 45:

\begin{tabular}{ | m{4cm} | m{4cm}| m{4cm} | m{2cm} | }
  \hline
    & Breast Cancer & No Cancer Diagnosis & Total\\ 
  \hline
  HCV Seropositive & 17 & 2 & 19 \\ 
  \hline
  HCV Seronegative & 110 & 91 & 201 \\ 
  \hline
  Total & 127 & 93 & 220 \\ 
  \hline
\end{tabular}

\vspace{1cm}

2x2 Contingency Table for patients older than 45:

\begin{tabular}{ | m{4cm} | m{4cm}| m{4cm} | m{2cm} | }
  \hline
    & Breast Cancer & No Cancer Diagnosos & Total\\ 
  \hline
  HCV Seropositive & 71 & 13 & 84 \\ 
  \hline
  HCV Seronegative & 207 & 39 & 246 \\ 
  \hline
  Total & 278 & 52 & 330 \\ 
  \hline
\end{tabular}


\section{Methods}
\label{sec:methods}

\begin{figure}[tbp]
  \centering
  \includegraphics[width=8cm]{table.png}
  \caption{2x2 Contingency Table}
  \label{fig:table}
\end{figure}

\begin{figure}[tbp]
  \centering
  \includegraphics[width=8cm]{formula.png}
  \caption{Hypergeometric pmf}
  \label{fig:formula}
\end{figure}

\begin{figure}[tbp]
  \centering
  \includegraphics[width=12cm]{2x2 stratified table.png}
  \caption{2x2 Contingency Table with Stratified Data}
  \label{fig:strat}
\end{figure}

\begin{figure}[tbp]
  \centering
  \includegraphics[width=8cm]{breslow day test stat.png}
  \caption{Breslow Day Test Statistic}
  \label{fig:BD}
\end{figure}

\begin{figure}[tbp]
  \centering
  \includegraphics[width=8cm]{MH Test Stat.png}
  \caption{Mantel-Haenszel Test Statistic}
  \label{fig:MH}
\end{figure}


To test the association of breast cancer and Hepatitis C seropositivity, one can use
the chi-square or Fisher's Exact test, depending on the
conditions from the sample size \citep{warner2013testing}. Fisher's
exact test is useful for when the normality assumption is violated 
and the expected values of the 2x2 table are too small. The test uses 
the hypergeometric distribution to test if the probabilities are
the same between the two groups. Thus, we can determine if there is 
more of a risk of breast cancer for those with seropositive Hep C.
In the data section, the data was summarized into 2x2 contingency tables.
The generalzied table is shown in Figure~\ref{fig:table}.
The first step is calculate the probability of the original table occuring,
using the hypergeometric pdf. Figure~\ref{fig:formula} shows the formula used.

Based on the alternative hypothesis, you then calculate the probability of 
all more extreme tables based on the a value. In this paper, p1 is the
probability of having breast cancer given that the patient is seropositive
and p2 is the probability of having breat cancer given that the patient
is seronegative. Then, we want to determine if there is an association,
specifically if the risk of having breast cancer is higher for those who
are seropositive. The alternative hypothesis is that p1 is greater than p2-
which will be tested in the application section.

After calculating the crude estimate, a Breslow-Day test will test for 
homogeneity of odds ratios between the age stratified data. The generalized
form of stratified 2x2 contingency tables is shown in Figure~\ref{fig:strat}.
The Breslow-Day test statistic is shown in Figure~\ref{fig:BD}, which will
be computed using SAS.

Based on the results of the Breslow Day test for homogeneity of odds ratios,
the Mantel-Haenszel test will compute a common odds ratio and test if there 
is an association between seropositivity and breast cancer while controlling 
for age. The Mantel-Haenszel test statistic is shown in Figure~\ref{fig:MH}
where the expected value of A is 
\begin{equation}
  E(A_{i})=\frac{M_{i1}R_{1i}}{N_{i}} 
\end{equation}
and the variance of A is 
\begin{equation}
  V(A_{i})=\frac{R_{1i}R_{0i}M_{1i}M_{0i}}{N_i^2(N_{i}-1)}. 
\end{equation}


\section{Application}
\label{sec:app}
-

\section{Discussion}
\label{sec:discuss}

-


\bibliography{../manuscript/refs}
\bibliographystyle{chicago}

\end{document}
