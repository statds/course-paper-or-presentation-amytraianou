\documentclass[12pt, titlepage]{article}

\usepackage{amsmath}
\usepackage[margin = 1in]{geometry}
\usepackage{graphicx}
\usepackage{booktabs}
\usepackage{natbib}
\usepackage{array}

\usepackage{lipsum}
\usepackage[colorlinks=true, citecolor=blue]{hyperref}


\title{Proposal: Association between Breast Cancer and Hepatitis C Virus}
\author{Amy Traianou\\
  Department of Statistics\\
  University of Connecticut
}

\begin{document}
\maketitle

\doublespace

\begin{abstract}
The Kolmogorov--Smirnov (KS) test is one of the most popular goodness-of-fit
tests for comparing a sample with a hypothesized continuous distribution.
Nevertheless, it has often been misused. The standard one-sample KS test applies
to independent, continuous data with a hypothesized distribution that is
completely specified. It is not uncommon, however, to see in the literature that
it was applied to dependent, discrete, or rounded data, with hypothesized
distributions containing estimated parameters. For example, it has been
``discovered'' multiple times that the test is too conservative when
hypothesized distribution has parameters that need to be estimated.
We demonstrate misuses of the one-sample KS test in
three scenarios through simulation studies:
1) the hypothesized distribution has unspecified parameters;
2) the data are serially dependent; and
3) a combination of the first two scenarios.
For each scenario, we provide remedies for practical applications.

\bigskip
\noindent{\sc Keywords}:
nonparametric bootstrap;
parametric bootstrap.
\end{abstract}


\paragraph{Introduction}
\label{sec:intro}
For this paper I decided to explore the risk factors associated with breast cancer in women. Breast cancer is extremely
prevalent and researchers are constantly trying to determine risk factors to identify women for preventative exams.
There are many studies attempting to determine if there is an association between certain risk factors and 
breast cancer including chronic hepatitis C infection \citep{Larrey2010is} and hepatitis B \citep{vishnu2016does}. 

Specifically, I want to determine if there is an association between testing positive for hepatitis C virus
and breast cancer. Both Hepatitis C and breast cancer are extremely prevalent in Egypt \citep{Hussein2021high}. 
Very recently on October 6th, Egypt and Qatar agreed to collaborate in the health sector and use each 
other's expertise. The health ministers specifically mentioned research in Hepatitis C and breast cancer
as a area of interest \citep{arham2022egypt}. Thus, research surrounding these diseases is becoming more
important in the field. 



\paragraph{Data}
\label{sec:data}
The data I will be using is from a retrospective case control study conducted in 2020 \citep{2020association}. The data includes 405 subjects as
part of the study group, all having been treated for breast cancer in the past 10 years. The second group consists of 145 adult females
from a governorate in Egypt, who all participated in a cross-sectional study from 2015-2017. This data can be put into a 2x2 contingency table
for statistical analysis. 

\vspace{1cm}

\begin{tabular}{ | m{5cm} | m{3cm}| m{3cm} | m{2cm} | }
  \hline
  Risk Factor & Study Group & Control & Total\\ 
  \hline
  Anti-HCV Seropositive & 88 & 15 & 103 \\ 
  \hline
  Anti-HCV Seronegative & 317 & 130 & 447 \\ 
  \hline
  Total & 405 & 145 & 550 \\ 
  \hline
  \label{table:total}
\end{tabular}

\begin{longtable}[]{@{}lll@{}}
\toprule
Risk Factor & Breast Cancer & Control & Total\tabularnewline
\midrule
\endhead
Anti-HCV Seropositive & 88 & 15 & 103\tabularnewline
Anti-HCV Seronegative & 317 & 130 & 447\tabularnewline
Total & 405 & 145 & 550\tabularnewline
\bottomrule
\caption{2x2 Contingency Table for All Patients}
\label{table:total}
\end{longtable}


\paragraph{Methods}
\label{sec:methods}
To test the association of breast cancer and Hepatitis C, I am going to use either the chi-square or Fisher's Exact test, depending on the
conditions from the sample size \citep{warner2013testing}. Fisher's exact test is useful for when the normality assumption is violated 
and the expected values of the 2x2 table are too small. The test uses the hypergeometric distribution to test if the probabilities are
the same between the two groups. Thus, we can determine if there is more of a risk of breast cancer for those with seropositive Hep C. As
 a result, we can determine if women who chronically test positive should be tested more frequently because they are at a higher risk. 
 
Figure~\ref{fig:formula} shows the probability of the original table occuring.

\begin{figure}[tbp]
  \centering
  \includegraphics[width=8cm]{formula.png}
  \caption{This is the hypergeometric pmf.}
  \label{fig:formula}
\end{figure}


\paragraph{Application}
\label{sec:app}
I expect to find an association between hepatitis C and breast cancer because there is some existing research that agrees with the association.
Therefore, this paper would corroborate the exisitng assumptions while giving them more of a basis. Although the potential impacts are quite 
minimal, the more research that supports the association, the better. If the investigation is not what I expect, then I would suggest more
data needs to be collected surrounding women with breast cancer in Eygpt. Specifically, a full prospective study with new samples. 


\paragraph{Discussion}
\label{sec:discuss}
I expect to find an association between hepatitis C and breast cancer because there is some existing research that agrees with the association.
Therefore, this paper would corroborate the exisitng assumptions while giving them more of a basis. Although the potential impacts are quite 
minimal, the more research that supports the association, the better. If the investigation is not what I expect, then I would suggest more
data needs to be collected surrounding women with breast cancer in Eygpt. Specifically, a full prospective study with new samples. 


\bibliography{../manuscript/refs}
\bibliographystyle{chicago}

\end{document}
