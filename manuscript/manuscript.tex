\documentclass[12pt, titlepage]{article}

\usepackage{amsmath}
\usepackage[margin = 1in]{geometry}
\usepackage{graphicx}
\usepackage{booktabs}
\usepackage{natbib}
\usepackage{array}

\usepackage{lipsum}
\usepackage[colorlinks=true, citecolor=blue]{hyperref}


\title{Association between Breast Cancer and Hepatitis C Virus with Age Controlling}
\author{Amy Traianou\\
  Department of Statistics\\
  University of Connecticut
}

\begin{document}
\maketitle


\begin{abstract}
In Egypt Hepatitus C virus infection is a common problem that often impedes treatment
of Breast Cancer in female patients. While previous work has analyze an associaiton 
between HCV seropositivity and breast cancer diagnosis, this paper will focus on the 
potential confounding of age on the association. We will use Fisher's Exact Test to
estimate the crude estimate of the association and the Mantel-Haenszel method to 
stratify the data between ages. The possible effect of age in the association can
lead to improved care for patients. 


\bigskip
\noindent{\sc Keywords}:
Breslow-Day;
Fisher's Exact Test;
Mantel-Haenszel.
\end{abstract}


\section{Introduction}
\label{sec:intro}

This paper will explore the risk factors associated with breast 
cancer in women. Breast cancer is extremely prevalent and researchers 
are constantly trying to determine risk factors to identify women for preventative exams.
There are many studies attempting to determine if there is an association 
between certain risk factors and breast cancer including chronic hepatitis 
C infection \citep{Larrey2010is} and hepatitis B \citep{vishnu2016does}. 

Specifically, this paper will examine if there is an association between 
testing positive for hepatitis C virus and breast cancer while controlling 
for age in Egyptian populations. Both Hepatitis C and breast 
cancer are extremely prevalent in Egypt \citep{Hussein2021high}.
Additionally, research has found that women who are seropositive for 
HCV have more agressive forms of breast cancer. However, there is 
minimal research concerning the effect of age on the associaiton. 

Recently on October 6th, Egypt and Qatar agreed to collaborate in the health sector and 
use each other's expertise. The health ministers specifically mentioned research 
in Hepatitis C and breast cancer as an area of interest \citep{arham2022egypt}. 
This meeting emphasizes the importance of diving into Hepatitus C Virus 
and it's association with breast cancer. There have been conflicting
foundings so studies should indenify additional variables that could 
contribute to better understanding the relationship. For that reason,
this paper is focusing on controlling for age of female patients. 

The rest of the paper is organized into several sections. Section~\ref{sec:data}
describes the study done in 2020 to test if HCV seropositivity is associated with 
breast cancer diagnosis in Egyptian population. Section~\ref{sec:methods} 
describes the following statistical methods; Fisher's Exact Test, Breslow-Day test,
and Mantel-Haenszel method. Section~\ref{sec:app} applies the methods to data 
from the study, including stratafied between ages. Section~\ref{sec:discuss} 
concludes the paper.



\section{Data}
\label{sec:data}

The data used in this paper is from a 2020 retrospective case-control 
study based on Egyptian female populations \citep{2020association},
sponsered by Mansoura University in Egypt.

The study group consists of 405 patients treated at the Oncology Center 
- Mansoura University in the past 10 years. In order to be included, 
invasive breast cancer must be biopsy-proven. Patients with unknown 
viral marker status, multiple cancer diagnoses and virus-unrelated 
hepatic pathology were all excluded. 

The control group consists of 
data from a previous study, conducted from 2015-2017. There are 145 
females from the same geographic distribution, all with no previous 
cancer diagnosis. In both groups, all patients are above the age of 
18. For this paper, patients are stratified based on age- older patients 
are considered 45 years or above and younger patients are below 45 years. 
The data is summarized in the following 2x2 contingency tables. 

\vspace{1cm}

\begin{minipage}{\textwidth}
2x2 Contingency Table for all patients:

\begin{tabular}{ | m{4cm} | m{4cm}| m{4cm} | m{2cm} | }
  \hline
    & Breast Cancer & No Cancer Diagnosis & Total\\ 
  \hline
  HCV Seropositive & 88 & 15 & 103 \\ 
  \hline
  HCV Seronegative & 317 & 130 & 447 \\ 
  \hline
  Total & 405 & 145 & 550 \\ 
  \hline
\end{tabular}

\vspace{1cm}

2x2 Contingency Table for pateints younger than 45:

\begin{tabular}{ | m{4cm} | m{4cm}| m{4cm} | m{2cm} | }
  \hline
    & Breast Cancer & No Cancer Diagnosis & Total\\ 
  \hline
  HCV Seropositive & 17 & 2 & 19 \\ 
  \hline
  HCV Seronegative & 110 & 91 & 201 \\ 
  \hline
  Total & 127 & 93 & 220 \\ 
  \hline
\end{tabular}

\vspace{1cm}

2x2 Contingency Table for patients older than 45:

\begin{tabular}{ | m{4cm} | m{4cm}| m{4cm} | m{2cm} | }
  \hline
    & Breast Cancer & No Cancer Diagnosos & Total\\ 
  \hline
  HCV Seropositive & 71 & 13 & 84 \\ 
  \hline
  HCV Seronegative & 207 & 39 & 246 \\ 
  \hline
  Total & 278 & 52 & 330 \\ 
  \hline
\end{tabular}

\vspace{1cm}

\end{minipage}


\section{Methods}
\label{sec:methods}

To test the association of breast cancer and Hepatitis C seropositivity, one can use
the chi-square or Fisher's Exact test, depending on the
conditions from the sample size \citep{warner2013testing}. Fisher's
exact test is useful for when the normality assumption is violated 
and the expected values of the 2x2 table are too small. The test uses 
the hypergeometric distribution to test if the probabilities are
the same between the two groups. Thus, we can determine if there is 
more of a risk of breast cancer for those with seropositive Hep C.
In the data section, the data was summarized into 2x2 contingency tables.
The generalzied table is shown in Figure~\ref{fig:table}.
The first step is calculate the probability of the original table occuring,
using the hypergeometric pdf:
\begin{equation}
P_a=[\frac{(a+b+c+d)a!b!c!d!}{(a+c)!(b+d)!(a+b)!(c+d)!}]^-1
\end{equation}

Based on the alternative hypothesis, you then calculate the probability of 
all more extreme tables based on the a value. In this paper, p1 is the
probability of having breast cancer given that the patient is seropositive
and p2 is the probability of having breat cancer given that the patient
is seronegative. Then, we want to determine if there is an association,
specifically if the risk of having breast cancer is higher for those who
are seropositive. The alternative hypothesis is that p1 is greater than p2-
which will be tested in the application section.

\begin{figure}[tbp]
  \centering
  \includegraphics[width=8cm]{table.png}
  \caption{2x2 Contingency Table}
  \label{fig:table}
\end{figure}

\begin{figure}[tbp]
  \centering
  \includegraphics[width=12cm]{2x2 stratified table.png}
  \caption{2x2 Contingency Table with Stratified Data}
  \label{fig:strat}
\end{figure}


After calculating the crude estimate, a Breslow-Day test will test for 
homogeneity of odds ratios between the age stratified data. The generalized
form of stratified 2x2 contingency tables is shown in Figure~\ref{fig:strat}.
The Breslow-Day test statistic will be computed using SAS with the formula:
\begin{equation}
  \chi_{BD}^2=\displaystyle\sum\limits_{i=0}^s\frac{[A_i-E(A_i|\hat{OR}_{M-H})]^2}{Var(A_i|\hat{OR}_{M-H})}
\end{equation}

Based on the results of the Breslow Day test for homogeneity of odds ratios,
the Mantel-Haenszel test will compute a common odds ratio and test if there 
is an association between seropositivity and breast cancer while controlling 
for age. The Mantel-Haenszel test statistic is:
\begin{equation}
  \chi_{M-H}^2=\frac{(|\sum\limits_{i=0}^s A_i - \sum\limits_{i=0}^sE(A_i)|-0.5)^2}
{\sum\limits_{i=0}^s V(A_i)}
\end{equation}

\vspace{1cm}

The expected value and variance of A for the ith strata is 
\[
  E(A_{i})=\frac{M_{i1}R_{1i}}{N_{i}} 
\]
\[
  V(A_{i})=\frac{R_{1i}R_{0i}M_{1i}M_{0i}}{N_i^2(N_{i}-1)}. 
\]


\section{Application}
\label{sec:app}
Fisher's exact test for the pooled data of all patients tested 
\[
H_{0}:p_{1}=p_{2}, H_{a}:p_{1}>p_{2}
\]
at a 0.05 level of confidence.

The test resulted in a p-value of 0.0027, which is less than 
alpha = 0.05. Thus, the null hypothesis is rejected and there 
is evidence suggesting patients who are seropositive have a
higher breast cancer rate than those who are seronegative.

Next, the Breslow Day test checks for homogeneity between the 
stratified ages. 
\[
H_{0}:OR_{1}=OR_{2}, H_{a}:OR_{1}\neq{OR_{2}}
\]
at a 0.01 level of confidence. 

The test resulted in a p-value of 0.0133, which is greater than
the alpha value. Thus, we can conclude the strata have similar
odds ratios and proceed with the Mantel-Haenszel procedure. 

The Mantel-Haenszel test produced a common odds ratio of 1.6847 and 
a 95 percent confidence interval of (0.9283, 3.0574). The interval
contains 1 so after controlling for age, the association is not supported.

However, the individual odds ratios for age groups are worth noting. For 
the older group (45 and older) the odds ratio estimate and interval are 
1.0290 and (0.5197, 2.0374). The odds ratio estimate and interval for the 
younger group (less than 45) are 7.0318 and (1.5828, 31.2399). Therefore,
the association between seropositivity and breast cancer is much stronger
for females less than 45 years. 

\section{Discussion}
\label{sec:discuss}
Based on the analysis, it is evident that for women under the age of 45,
seropositivity and breast cancer are strongly associated. As being HCV 
positive can complicate treatment of cancer, this is critical information.
In Egypt, breast cancer remains the most common form or cancer and high 
levels of hepatitus C continue. This study was retrospective so analysis 
was limited to observing odds ratios and the sample size was quite small.
Thus, future studies should aim to be prospective cohort studies so 
researchers can observe how seropisitivity and breast cancer interact 
over time. 
Additionally, age is just one factor that can affect the associaiton 
between hepatitus C and breast cancer. Future studies should dive 
deeper into more factors such as weight, reproductive history, 
and alcohol consumption. Addiitonally, future studies can explore how 
seropositivity affects tumor size, disease agression, and more. As 
breast cancer is so prevalent in Egypt, further understanding the 
risk factors is becoming even more importnant. 


\bibliography{../manuscript/refs}
\bibliographystyle{chicago}

\end{document}
