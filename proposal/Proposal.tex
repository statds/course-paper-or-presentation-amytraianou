\documentclass[12pt]{article}

\usepackage{amsmath}
\usepackage[margin = 1in]{geometry}
\usepackage{graphicx}
\usepackage{booktabs}
\usepackage{natbib}

\usepackage{lipsum}
\usepackage[colorlinks=true, citecolor=blue]{hyperref}


%% meta data

\title{Proposal: Association between Breast Cancer and Hepatitis C Virus}
\author{Amy Traianou\\
  Department of Statistics\\
  University of Connecticut
}

\begin{document}
\maketitle


\paragraph{Introduction}
For this paper I decided to explore the risk factors associated with breast cancer in women. 
\citep{dwivedi2017analysis}


\paragraph{Specific Aims}
Specifically, I want to determine if there is an association between testing positive for hepatitis C virus
and breast cancer. Both Hepatitis C and breast cancer are extremely prevalent in Egypt \citep{Hussein2021high}. 
Very recently on October 6th, Egypt and Qatar agreed to collaborate in the health sector and use each 
other's expertise. The health ministers specifically mentioned research in Hepatitis C and breast cancer
as a area of interest \citep{arham2022egypt}. Thus, research surrounding these diseases is becoming more
important in the field. 

\paragraph{Data}
The data I will be using is from a retrospective study conducted in 2020 \citep{2020association}. The data includes 405 subjects as
part of the study group, all having been treated for breast cancer in the past 10 years. The second group consists of 145 adult females
from a governorate in Egypt, who all participated in a cross-sectional study from 2015-2017.

\paragraph{Research Design and Methods}
To test the association of breast cancer and Hepatitis C, I am going to use Fisher's Exact test. 

\paragraph{Discussion}
I expect to find an association. 

\paragraph{Conclusion}
This is the conclusion. 



\bibliography{../manuscript/refs}
\bibliographystyle{chicago}

\end{document}
